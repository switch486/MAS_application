Rozpoznawanie obrazów jest jednym z najważniejszych zastosowań sztucznej inteligencji. Właściwie, aby można było mówić o sztucznej inteligencji konieczne jest uwzględnienie zaawansowanych technik rozpoznawania obrazów. Mimo, że zdecydowana większość problemów sztucznej inteligencji jest problemami trudnymi\footnote{W literaturze angielskojęzycznej określa się to pojęciem AI-complete.}, to wydaje się, że na równi z rozpoznawaniem języka, jako podstawowe należy uznawać rozpoznawanie obrazów.\\

W tej pracy przedstawione będą podstawowe techniki stosowane w rozpoznawaniu obrazów, które w idei mają być komputerowym odpowiednikiem kory wzrokowej. Techniki te są mniej lub bardziej związane z dziedzinami rozpoznawania obrazów w grafice komputerowej i naukami o mózgu, które wcześniej wspomnianej kory wzrokowej dotyczą. Można by pokusić się o narysowanie odcinka na płaszczyźnie, którego końcami byłby te dwie dziedziny. Wtedy wszystkie istniejące projekty dotyczące rozpoznawania obrazów można by umieścić na tej płaszczyźnie. Były by one umiejscowione na tym odcinku w pewnej odległości, która określałaby powiązanie z konkretną dziedziną. Celem tej pracy jest znalezienie tego najlepszego punktu na tym odcinku, który będzie wykorzystywać zaawansowane techniki z dziedziny rozpoznawania obrazów i te zaobserwowane w korze wzrokowej. Zagadnienie to zostało sprowadzone do jednego wymiaru celowo, żeby pokazać skąd należy czerpać inspiracje, faktem jest jednak, że ciężko jest wyrazić złożoność tematu w jednym wymiarze. Aby lepiej przedstawić to co się dzieje w dziedzinie należałoby zdaniem autora skupić się na korze wzrokowej i dzieląc ją na elementy pokazać co, gdzie i w jaki sposób jest przedstawiane.
